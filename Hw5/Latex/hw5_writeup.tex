\documentclass[12pt]{article}
\usepackage[a4paper,margin=.5in]{geometry}
\usepackage{graphicx}
\usepackage{booktabs}
\usepackage{listings}
\usepackage{color}

\definecolor{dkgreen}{rgb}{0,0.6,0}
\definecolor{gray}{rgb}{0.5,0.5,0.5}
\definecolor{mauve}{rgb}{0.58,0,0.82}

\lstset{frame=tb,
  language=Python,
  aboveskip=3mm,
  belowskip=3mm,
  showstringspaces=false,
  columns=flexible,
  basicstyle={\small\ttfamily},
  numbers=none,
  numberstyle=\tiny\color{gray},
  keywordstyle=\color{blue},
  commentstyle=\color{dkgreen},
  stringstyle=\color{mauve},
  breaklines=true,
  breakatwhitespace=true,
  tabsize=3
}
%\usepackage{subfig}
\usepackage{subcaption}
\usepackage{hyperref}
\hypersetup{
    colorlinks=true,
    linkcolor=blue,
    filecolor=magenta,      
    urlcolor=cyan,
    pdftitle={Overleaf Example},
    pdfpagemode=FullScreen,
    }
\newcommand*{\figuretitle}[1]{%
    {\centering%   <--------  will only affect the title because of the grouping (by the
    \textbf{#1}%              braces before \centering and behind \medskip). If you remove
    \par\medskip}%            these braces the whole body of a {figure} env will be centered.
}
\title{Homework 5}

\author{Tylman Michael\\CSE 546 Machine Learning}
\date{4/5/2023}
%moderncv theme
\usepackage[utf8]{inputenc} 
\begin{document}
\maketitle{}
\section{Problem 1:}
\subsection{Part a:}
For problem 1a, we were tasked with finding the optimal number of clusters by finding the elbow point. Now, since I have 
a mathematics and physics background, I couldn't simply be happy with eyeballing it. So, I did a little bit of research
on how to mathematically find an elbow point, and I found an interesting paper on detecting knee points in system behavior
which I will link to \href{https://raghavan.usc.edu//papers/kneedle-simplex11.pdf}{HERE}. The algorithm represented 
in this paper is implemented in the kneed package in python, and it is what I used to automatically detect the optimal 
number of clusters.

The SSE vs. No. clusters plot can be seen in the first part of Figure \ref{figure1}, where we can see the point decided 
by the kneed method marked as a red vertical line.
\begin{figure}
    \begin{subfigure}{.5\textwidth}
        \includegraphics[width=.95\textwidth]{../results/kmeans/SSE_Cluster_Plot.png}
        \caption{SSE vs. No. Clusters}
        \end{subfigure}%
      \begin{subfigure}{.5\textwidth}
        \includegraphics[width=.95\textwidth]{../results/kmeans/Silhouette_Plot.png}
        \caption{Silhouette Plot}
      \end{subfigure}
      \label{figure1}
\end{figure}

\subsection{Part b:}
For problem 1b, we were tasked with generating the Silhouette plot for the optimal choice of clusters. This plot is 
shown in the second part of Figure \ref{figure1}. In this plot we can see that cluster number 1 performed the very 
best, with no elements past 0. However, we can still see that the overall average Silhouette score is quite low, 
clocking in at less than .10 with  some clusters (cluster 2 in particular) showing non-trivial amounts of negative
values.

\subsection{Part c:}
For problem 1c, we were tasked with finding images at the core of their clusters and on the edge of their clusters.
To find the images at the core of their clusters, I took the top 5 highest scoring sample from each cluster. To find
the images at the edge of their clusters, I took the bottom 2 lowest absolute value samples from each cluster. All clusters
except for number 1 had members very close to 0, but cluster 1 had it's lowest members hovering around .2 which was orders
of magnitude higher than the other clusters. Given that behavior, I decided to not include cluster 1 results in the analysis 
of the edge cases given the stipulation of the assignment to only include border cases if they exist.

Looking at the images at the center of clusters, it's clear that the clusters are boats, cars, birds, horses, and planes, 
respectively. This will be important to keep in mind as we analyze the images at the edge of clusters in Figure \ref{figure3}.


Right away in cluster 2 (cars), we can see that these images are predominantly green and sky blue, respectively. I think that these
backgrounds biased the values to appear more like horses and planes than they would normally seem. Additionally, the car
on the right is a metallic color, which is unlike other cars in its class that are usually painted but more typical of 
planes. I think it is reasonable that these images are bordering two clusters, even though they were correctly clustered.

In cluster 3 (birds), we can see that the image of an ostrich was barely clustered with the birds. I think this is a 
totally fair way to treat the ostrich. If there were any creature I'd put on the border of birds and horses, it would be 
the ostrich. The next image on the border is a car which was incorrectly clustered as a bird, but notably it's the first 
image of a car I've come across which is both facing directly into the camera, and has a door open. Additionally, this 
car is in a field and does not have colors which are not readily found in nature. With these facts in mind, I think that 
this picture is likely reasonable for miss-clustering, but would warrant an analysis of the feature space to understand
what features exactly caused this error.

In cluster 4 (horses), we can see that the first image has a smaller horse which is the darkest color in the image, along
with having it's tail up. Additionally, the front legs are a bit blurred, which causes the image to evoke a similar shape 
and coloration to a rooster nearby a barn. I think that these factors caused this image to be placed on the edge of the 
horse cluster. The right image of the horse cluster is what I believe to be the most intriguing image in this analysis.
This image appears to clearly be a horse, and the only features I can find that could cause it to struggle are the 
inclusion of the person, and the tri-layer background of alternating light-green-dark that was also present in the left 
horse image. 

Finally, in cluster 5 (planes), we can see that the first image is of a plane taken at the same altitude as plane. This 
angle only leaves the tail fin and the cockpit as defining features, and most importantly removes any information 
about the plane's wings. I think that perfectly symmetric wings angled towards a point is likely one of the key differentiating features of 
a plane vs. a boat, given their similar blue shifted backgrounds lacking foliage. On the right image we see a duck 
resting in choppy water. Most importantly, we can see the ducks reflection in the water, which is giving it a symmetric shape
(the heads) angled towards a point (the tail). I think this symmetry along with the cloud-colored water is what caused this 
image to be clusterd on the edge of the plane cluster.


\begin{figure}
    \includegraphics{../results/kmeans/Center_Kmeans_images.png}
    \caption{Images at the Center of Clusters Kmeans}
    \label{figure2}
\end{figure}
\begin{figure}
    \includegraphics{../results/kmeans/Border_Kmeans_images.png}
    \caption{Images at the Edge of Clusters Kmeans}
    \label{figure3}
\end{figure}

\section{Problem 2:}
\subsection{Part a:}
When it came time to analyze the dendrogram to select the optimal amount of clusters, it was difficult to remove my 
inherent bias towards selecting 5 clusters. Given the fact that the KMeans gave 5 clusters and the ground truth had 5 
classes I had strong prompting to select 5 clusters again. But I wanted to look at this through an objective lens and 
find a feature of the dendrogram which could justify my choice. 

After spending some time looking at Figure \ref{figure4}, I realized that visually the optimal choice of 5 clusters 
minimized the ratio of the variance of the cluster distance over the average cluster distance weighted by the inverse
of the number of
clusters. This is my first interaction with a dendrogram for picking number of clusters, so I was unable to develop and 
test an automated function which matches my hypothesis, and I could not find online others who may have calculated this 
in the same way. Likely, the fact that this method is not easily findable online means that my hypothesis is flawed and
may be overly influence by the behavior of this toy dataset which is well-balanced. Regardless, it is the reasoning which 
I found to best capture why I believe the dendrogram shows 5 clusters to be the optimal number of clusters for the ward method.
\begin{figure}
    \includegraphics{../results/agglo/Warddendrogram.png}
    \caption{Ward Dendrogram}
    \label{figure4}
\end{figure}

\subsection{Part b:}
The Silhouette plot for the ward method is in Figure \ref{figure5}. This figure is evidence that the agglomerative method
may not be better than the KMeans algorithm for this dataset. Previously, the clusters showed far fewer amounts of points
with a negative Silhouette score, even at a similar overall average score. More in-depth discussion is to follow in part c:
\begin{figure}
    \includegraphics{../results/agglo/Silhouette_Plot_ward.png}
    \caption{Ward Silhouette}
    \label{figure5}
\end{figure}

\subsection{Part c:}
The repeated plots of parts a and b for single linkage and complete linkage are shown in figures \ref{figure6} and \ref{figure7}, 
respectively.

Starting off with the odd one out, the single linkage clustering algorithm appears to have completely missed the mark. 
It seems to have piece-by-piece grown one cluster a single sample at a time until it encompassed the whole group. This 
plot and it's accompanying Silhouette plot show beyond a shadow of a doubt that single linkage is not the direction 
to go with this data. Even when I forced the number of clusters to be 5, the single linkage clustering method caused 
every item to be placed in one group, and no items to be placed in any other clusters leaving 4 empty clusters.

Next, the complete clustering algorithm appears much more reasonable. In an attempt to be fair, I am going to try and 
eyeball my same reasoning I used before. Looking around the line provided in the image, we can see that it appears like 
6 clusters are the optimal number of clusters for this algorithm. However, we can also see that the variance between
the clusters (and cluster balance, but that must imply a balanced dataset) is much worse than the ward algorithm managed.
in the Silhouetteplot, we can see that cluster 5 is nothing but an extremely tiny sliver, and should likely be merged
in with another cluster.
\begin{figure}
    \begin{subfigure}{.5\textwidth}
        \includegraphics[width=.95\textwidth]{../results/agglo/Singledendrogram.png}
        \caption{Single Dendrogram}
        \end{subfigure}%
      \begin{subfigure}{.5\textwidth}
        \includegraphics[width=.95\textwidth]{../results/agglo/Completedendrogram.png}
        \caption{Complete Dendrogram}
      \end{subfigure}
      \label{figure6}
\end{figure}
\begin{figure}
    \begin{subfigure}{.5\textwidth}
        \includegraphics[width=.95\textwidth]{../results/agglo/Silhouette_Plot_single.png}
        \caption{Single Silhouette}
        \end{subfigure}%
      \begin{subfigure}{.5\textwidth}
        \includegraphics[width=.95\textwidth]{../results/agglo/Silhouette_Plot_complete.png}
        \caption{Complete Silhouette}
      \end{subfigure}
      \label{figure7}
\end{figure}


\end{document}